\documentclass{article}
\title{Funktionale Programmierung Mitschrieb}
\usepackage{fancyhdr}
\pagestyle{fancy}
\usepackage{enumerate}
\usepackage{amsmath}
\usepackage{amsfonts}
\usepackage{amssymb}
\fancyhf{}
\fancyhead[C]{Torsten Grust - Functional Programming}
\usepackage{fontspec}
\setmonofont{Andale Mono}
\usepackage[english]{babel}
\author{Finn Ickler}
\usepackage{epigraph}
\usepackage{minted}
\renewcommand{\listingscaption}{Codebeispiel}
\newcommand{\Haskell}[1]{\mintinline{Haskell}{#1}}
\setcounter{secnumdepth}{-1}
\begin{document}
\maketitle
\epigraph{\glqq Avoid sucess at all cost \grqq}{Simon Peyton Jones}
\listoflistings
\section{Vorlesung 1}
\begin{listing}
\caption{Hello World}
\begin{minted}{haskell}
-- Hello World Haskell
main :: IO ()
main = putStrLn "Chewie, we're home"
\end{minted}
\end{listing}
\subsection{Functional Programming (FP)}
A programming language is a medium for expressive ideas (not to get a computer to perform operations ). Thus programs must be written for people to read, and only incidentally for machines.
\subsection{Computational Model in FP : \emph{Reduction}}
Replace expressions by their value.\\
IN FP, expressions are formed by applying functions to values.
\begin{enumerate}
\item Function as in maths: $x = y \rightarrow f(x) = f(y)$
\item Functions are values like numbers or text
\end{enumerate}
\begin{tabular}{l|c|c}
&FP&Imperative\\
construction & function application and composition & statement sequencing\\
execution & reduction (expression evaluation) & state changes\\
sementics & $\lambda$-calculus&denotational
\end{tabular}\bigskip\\
$ n \in \mathbb{N}, n \geq 2$ is a prime number $\Leftrightarrow$ the set of non-trivial factors of n is empty.\\
$n$ is prime $\Leftrightarrow \{ m \mid m \in m \in \{2,\ldots,n-1\}, n mod m = 0 \} = \{\}$\\
\begin{listing}[h!]
\begin{minted}{c}
int IsPrime(int n)
{
    int m;
    int found_factor;
    found_factor
    for (m = 2; m <= n -1; m++)
    {
        if (n % m == 0)
        {
            found_factor = 1 ;
            break;
        }
    }
    return !found_factor;
}
\end{minted}
\caption{isPrime in C}
\end{listing}
\begin{listing}[h!]
\begin{minted}{Haskell}
isPrime :: Integer -> Bool
isPrime n = factors n == []
  where 
    factors :: Integer -> [Integer]
    factors n = [ m  | m <- [2..n-1], mod n m == 0]

main :: IO ()
main = do
  let n = 42
  print (isPrime n)
\end{minted}
\caption{isPrime in Haskell}
\end{listing}
\newpage
\usemintedstyle[Haskell]{bw}
\definecolor{bg}{rgb}{0.95,0.95,0.95}
\begin{listing}[h!]
\begin{minted}[bgcolor=bg]{Haskell}
let xs = [ x+1 | x <- [0..9] ]
:sprint xs = _
length xs
:sprint xs = [_,_,_,_,_,_,_,_,_]
\end{minted}
\caption{Lazy Evaluation in der ghci REPL}
\end{listing}
\usemintedstyle[Haskell]{default}
\subsection{Haskell Ramp Up}
Read $\equiv$ as ''denotes the same value as''\\
Apply f to value e:\quad f \textvisiblespace e \\(juxtaposition, ''apply'', binary operator \textvisiblespace, Haskell speak: infixL 10 \textvisiblespace)
= \textvisiblespace has max precedere (10): f $e_1$ +$e_2 \equiv $(f $e_1$) + $e_2$
\textvisiblespace associates to the left g \textvisiblespace f \textvisiblespace e $\equiv$ (g f) e
Function composition:
\begin{enumerate}[-]
\item g (f e)
\item Operator ''.'' (''after'') : (g.f) e (. = $\circ$) = g(f (e))
\item Alternative ''apply'' operator \$ (lowest precedure, associates to the right), infix 0\$): f\$$e_1$+ $e_2$ = f ($e_1 + e_2$)
\end{enumerate}
\section{Vorlesung 2}
\begin{listing}
\caption{Verschiedene Schreibweise einer Applikation}
\begin{minted}[bgcolor=bg]{Haskell}
cos 2 * pi
cos (2 * pi)
cos $ 2 * pi
isLetter (head (reverse ("It's a " ++ "Trap")))
(isLetter . head . reverse ) ("It's a" ++ "Trap")
isLetter $ head $ reverse $ "It's a" ++ "Trap"
\end{minted}
\end{listing}
\noindent Prefix application of binary infix operator $\oplus$
\begin{minted}[escapeinside=||]{Haskell}
(|$\oplus) e_1 e_2 \equiv e_1 \oplus e_2$|
|$(\&\&)$| True False |$\equiv$| False
\end{minted}
Infix application of binary function f:
\begin{minted}[escapeinside=||]{Haskell}
|$e_1$| `f` |$e_2$| |$\equiv$| f |$e_1 e_2$|
x `elem` xs |$\equiv$| x |$\in$| xs
\end{minted}
User defined operators with characters : \mintinline{Haskell}|!#%&*+/<=>?@\^|~|
\begin{listing}
\caption{Eigener $\approx$ Opperator}
\begin{minted}{Haskell}
epsilon :: Double
epsilon = 0.00001
(~=~) :: Double -> Double -> Bool
x ~=~ y = abs (x - y) < epsilon
infix 4 ~=~ 
\end{minted}
\end{listing}
\subsection{Values and Types}
Read \mintinline{Haskell}|::| as ''has type''\\
Any Haskell value e has a type t (\mintinline{Haskell}{e::t}) that is determined at compile time.\\
The \mintinline{Haskell}|::| type assignement is either given explicitly or inferred by the computer
\subsection{Types}
\begin{tabular}{lll}
Type&Description&Value\\
\hline
Int & fixed precision integers ($-2^{63}\ldots2^{63}-1$)&\Haskell{0,1,42}\\
Integer & arbitrary Precision integers & \Haskell{0,10^100}\\
Float,Double & Single/Double precision floating points & \Haskell{0.1,1e03}\\
Char & Unicode Character&\mintinline[escapeinside=||]{Haskell}{'x','\t', '△', '\8710'}\\
Bool & Booleans & \mintinline[escapeinside=||]{Haskell}{True, False}\\
() & Unit (single-value type) & \mintinline[escapeinside=||]{Haskell}{()}
\end{tabular}
\begin{minted}[bgcolor =bg]{Haskell}
2
it :: Integer
42 :: Int 
it :: Int
'a' 
it :: Char
True 
it :: Bool
10^100 
it :: Integer
10^100 :: Double 
it :: Double
\end{minted}
\subsection{Type Constructors}
\begin{itemize}
\item Build new types from existing Types
\item Let a,b denote arbitrary Types (type variables)
\end{itemize}
\begin{tabular}{lp{6cm}l}
Type Constructor& Description & Values\\
\hline
(a,b)&pairs of values of types a and b& \Haskell{(1,True) :: (Int, Bool)}\\
($\text{a}_1,\text{a}_2,\ldots,\text{a}_n$)& n-Types& \Haskell{2,False :: (Int, Bool)}\\
\lbrack a\rbrack &list of values of type a & \Haskell{[] :: [a]}\\
\Haskell{Maybe} a & optional value of type a & \Haskell{Just 42 Maybe Integer}\\
& & \Haskell{Nothing :: Maybe a}\\
\Haskell{Either} a b & Choice between values of Type a and b& \Haskell{Left 'x' :: Either Char b}\\
& & \Haskell{Right pi :: Either a Double}\\
\Haskell{IO} a & I/O action that returns a value of type a (can habe side effects ) & \Haskell{print 42 :: IO} ()\\
& & \Haskell{getChar :: IO Char}\\
a \Haskell{->} b & function from type a to b & \Haskell{isLetter :: Char -> Bool}
\end{tabular}
\begin{minted}[bgcolor=bg]{Haskell}
(1, '1', 1.0)
it :: (Integer, Char, Double)
[1, '1', 1.0]
it :: Fehler
[0.1,1.0,0.01] 
it :: [Double]
[]
it :: [t]
"Yoda"
it :: [Char]
['Y', 'o', 'd', 'a']
"Yoda"
[Just 0, Nothing, Just 2] 
it :: [Maybe Integer]
[Left True, Right 'a']
it :: [Either Bool Char]
print 'x' 
it :: ()
getChar
*
it :: Char
:t getChar
getChar :: Io Char
:t fst 
fst :: (a,b) -> a
:t snd
snd :: (a,b) -> b
:t head
head :: [a] -> a
:t (++)
(++) :: [a] -> [a] -> [a]
\end{minted}
\subsection{Currying}
\begin{itemize}
\item Recall:\begin{enumerate}
\item $e_1$ \Haskell{++} $e_2 \equiv$ \Haskell{(++)} $e_1 e_2$
\item \Haskell{++} $e_1 e_2 \equiv$ \Haskell{((++)} $e_1) e_2$
\end{enumerate}
\item Function application happens one argument at a time (currying, Haskell B. Curry)
\item Type of n-ary function: : $a_1$ \Haskell{->} $a_2$ \Haskell{... ->} $a_n$ \Haskell{-> b}
\item Type constructor \Haskell{->} associates to the right thus read the type as:\\ $a_1$ \Haskell{->} ($a_2$ \Haskell{->} $a_3$ (\Haskell{... ->} ($a_n$ \Haskell{-> b})\ldots))
\item Enables partial application: ''Give me a value of type $a_1$, I'll give you a (n-1)-ary function of type $a_2$ \Haskell{->} $a_3$ \Haskell{-> ... ->} $a_n$ \Haskell{-> b}
\end{itemize}
\begin{minted}[bgcolor = bg]{Haskell}
"Chew" ++ "bacca"
"Chewbacca"
(++) "Chew" "bacca"
"Chewbacca"
((++) "Chew") "bacca"
"Chewbacca"
:t (++) "Chew"
"Chew" :: [Char] -> [Char]
let chew = (++) "Chew"
chew "bacca"
"Chewbacca"
let double (*) 2
double 21
42
\end{minted}
\subsection{Defining Values (and thus: Functions)}
\begin{itemize}
\item = binds naems to values, names must not start with A-Z (Haskell style: camelCase)
\end{itemize}
\end{document}